
%######################################### Packages and new functions definitions ###########################################%
\documentclass[11pt]{sample}
\usepackage{pslatex}
\usepackage[T1]{fontenc}
\usepackage[utf8]{inputenc}
\usepackage{CJK}
\newcommand\japonais[1]{\begin{CJK*}{UTF8}{song}#1\end{CJK*}}
\usepackage{fancybox}
\usepackage{a4wide}
\usepackage[toc]{appendix}
\usepackage{color}
\usepackage{algorithm}
\usepackage{algorithmic}
\usepackage{graphicx}
\usepackage{graphics}

\graphicspath{{./images/}}

\bibliographystyle{prsty}

\renewcommand{\algorithmicrequire} {\textbf{\textsc{Entrées:}}}
\renewcommand{\algorithmicensure}  {\textbf{\textsc{Sorties:}}}
\renewcommand{\algorithmicwhile}   {\textbf{Tant Que}}
\renewcommand{\algorithmicdo}      {\textbf{faire}}
\renewcommand{\algorithmicendwhile}{\textbf{Fin tant que}}
\renewcommand{\algorithmicend}     {\textbf{Fin}}
\renewcommand{\algorithmicif}      {\textbf{Si}}
\renewcommand{\algorithmicendif}   {\textbf{Fin Si}}
\renewcommand{\algorithmicelse}    {\textbf{Sinon}}
\renewcommand{\algorithmicthen}    {\textbf{Alors}}
\renewcommand{\algorithmicfor}     {\textbf{Pour}}
\renewcommand{\algorithmicforall}  {\textbf{Pour tout}}
\renewcommand{\algorithmicdo}      {\textbf{Faire}}
\renewcommand{\algorithmicendfor}  {\textbf{Fin pour}}
\renewcommand{\algorithmicloop}    {\textbf{Boucler}}
\renewcommand{\algorithmicendloop} {\textbf{Fin boucle}}
\renewcommand{\algorithmicrepeat}  {\textbf{Répéter}}
\renewcommand{\algorithmicuntil}   {\textbf{Jusqu'à}}

\floatname{algorithm}{Algorithme}

\let\mylistof\listof
\renewcommand\listof[2]{\mylistof{algorithm}{List of algorithm}}

\makeatletter
\providecommand*{\toclevel@algorithm}{0}
\makeatother
\makeatletter
\def\clap#1{\hbox to 0pt{\hss #1\hss}}%
\def\ligne#1{%
  \hbox to \hsize{%
    \vbox{\centering #1}}}%
\def\haut#1#2#3#4{%
  \hbox to \hsize{%
    \rlap{\vtop{\centering #1}}%
    \rlap{\vtop{\centering #2}}%
    \clap{\vtop{\centering #3}}%
    \llap{\vtop{\centering #4}}}}%
\def\bas#1#2#3{%
  \hbox to \hsize{%
    \rlap{\vbox{\raggedright #1}}%
    \hss
    \clap{\vbox{\centering #2}}%
    \hss
    \llap{\vbox{\raggedleft #3}}}}%
\def\maketitle{%
  \thispagestyle{empty}\vbox to \vsize{%
    \haut{}{\@blurb}{}
    \vfill
    \vspace{4cm}
    \ligne{\Large \@title}
    \vspace{5mm}
    \ligne{\Large \@author}
       \vspace{1cm}
    \vfill
    \vfill
    \bas{}{\@location, \@date\newline}{}
    }%
  \cleardoublepage
  }
\def\date#1{\def\@date{#1}}
\def\author#1{\def\@author{#1}}
\def\title#1{\def\@title{#1}}
\def\location#1{\def\@location{#1}}
\def\blurb#1{\def\@blurb{#1}}
\def\email#1{\def\@email{#1}}
\def\logo#1{\def\includegraphics[height=3\baselineskip]{#1}}
\date{\today}
\author{}
\title{}
\location{\japonais{一関高専}}
\blurb{}
\email{steeve.vandecappelle@free.fr}
\makeatother
  \title{\begin{Huge}\textcolor{blue}{\japonais{一万円}}\end{Huge}}
  \author{Vandecappelle  \textsc{Steeve} }
  \date{\japonais{2009年 四月から 六月まで}}
  \location{\japonais{一関高専}}
  %\includegraphics[height=3\baselineskip]{iutaustl.jpg}
  \blurb{%
  %\includegraphics[height=3\baselineskip]{iutaustl.jpg}\\
  \hspace{1mm}\includegraphics[height=3\baselineskip]{iutaustl.jpg}\hspace{100mm}\includegraphics[height=3\baselineskip]{ichinoseki-koosen.jpg}\\
    Insititut Universitaire de Lille 1 \\
    \japonais{一関高専} \\[1em]
    \japonais{研修の レポート}\\
    \japonais{小保方先生}\\
  }%
  
  
  
  
  
  
  
  %####################################################### Beginning document #########################################################%
\begin{document}
\maketitle
\newpage
\strut
\newpage
\startcontents[sections]
%########################################################## Contents ############################################################%
\section*{\japonais{綱領}} 
\printcontents[sections]{l}{1}{\setcounter{tocdepth}{2}}
\newpage

%####################################################### Thanks Section #########################################################%
\section*{\japonais{礼状}\markboth{thanks}{thanks}}
\addcontentsline{toc}{section}{\protect\numberline{}thanks} 
\japonais{まず 私は 一関高専の こうちょうに おれい を いいたい。 そして 私の チューター 小保方先生 
\\みなさんの先生が 私を たすけてくれました。
\\それに 私は 5Sの 学生に おれい を いいたい。
}




\newpage
%###################################################### Abstract Section ########################################################%
\section*{\japonais{抄録}\markboth{Abstract}{Abstract}}
\addcontentsline{toc}{section}{\protect\numberline{}Abstract} 
\japonais{この けんしゅうは 私の二年のだいがくの まとめです。\\
私の けんきゅうに 一関の先生は 期待していました。
\\
この けんしゅうの さいしゅう もくてきは グラフィック アプリケーションの 一万円ゲム 作りました。
私は この  アプリケーションの プログラムを 考えました


}



\newpage
%###################################################### Documents Section #######################################################%
\section*{\japonais{付録}\markboth{Aditional documents}{Aditional documents}}
\addcontentsline{toc}{section}{\protect\numberline{}Aditional documents} 
\appendix
\newpage


%################################################### Extern sources Section #####################################################%
\section*{\japonais{リンックと}\markboth{links}{links}}
\addcontentsline{toc}{section}{\protect\numberline{}links} 
\japonais{日仏辞典:} http://www.dictionnaire-japonais.com/






\end{document}
